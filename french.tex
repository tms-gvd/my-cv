%-------------------------
% Resume in Latex
% Author : Aras Gungore
% License : MIT
%------------------------

\documentclass[letterpaper,11pt]{article}

\usepackage{latexsym}
\usepackage[empty]{fullpage}
\usepackage{titlesec}
\usepackage{marvosym}
\usepackage[usenames,dvipsnames]{color}
\usepackage{verbatim}
\usepackage{enumitem}
\usepackage[hidelinks]{hyperref}
\usepackage{fancyhdr}
\usepackage[english]{babel}
\usepackage{tabularx}
\usepackage{hyphenat}
\usepackage{fontawesome}
% \input{glyphtounicode}


%---------- FONT OPTIONS ----------
% sans-serif
% \usepackage[sfdefault]{FiraSans}
% \usepackage[sfdefault]{roboto}
% \usepackage[sfdefault]{noto-sans}
% \usepackage[default]{sourcesanspro}

% serif
% \usepackage{CormorantGaramond}
% \usepackage{charter}


\pagestyle{fancy}
\fancyhf{} % clear all header and footer fields
\fancyfoot{}
\renewcommand{\headrulewidth}{0pt}
\renewcommand{\footrulewidth}{0pt}
\setlength{\footskip}{4.1pt}

% Adjust margins
\addtolength{\oddsidemargin}{-0.5in}
\addtolength{\evensidemargin}{-0.5in}
\addtolength{\textwidth}{1in}
\addtolength{\topmargin}{-.5in}
\addtolength{\textheight}{1.0in}

\urlstyle{same}

\raggedbottom
\raggedright
\setlength{\tabcolsep}{0in}

% Sections formatting
\titleformat{\section}{
  \vspace{-4pt}\scshape\raggedright\large
}{}{0em}{}[\color{black}\titlerule \vspace{-5pt}]

% Ensure that generate pdf is machine readable/ATS parsable
\pdfgentounicode=1

%-------------------------
% Custom commands

\newcommand{\resumeItem}[1]{
  \item\small{
    {#1 \vspace{-2pt}}
  }
}


\newcommand{\resumeSubheading}[4]{
  \vspace{-2pt}\item
    \begin{tabular*}{0.97\textwidth}[t]{l@{\extracolsep{\fill}}r}
      \textbf{#1} & #2 \\
      \textit{\small#3} & \textit{\small #4} \\
    \end{tabular*}\vspace{-7pt}
}


\newcommand{\resumeSubSubheading}[2]{
    \vspace{-2pt}\item
    \begin{tabular*}{0.97\textwidth}{l@{\extracolsep{\fill}}r}
      \textit{\small#1} & \textit{\small #2} \\
    \end{tabular*}\vspace{-7pt}
}


\newcommand{\resumeEducationHeading}[6]{
  \vspace{-2pt}\item
    \begin{tabular*}{0.97\textwidth}[t]{l@{\extracolsep{\fill}}r}
      \textbf{#1} & #2 \\
      \textit{\small#3} & \textit{\small #4} \\
      \textit{\small#5} & \textit{\small #6} \\
    \end{tabular*}\vspace{-5pt}
}


\newcommand{\resumeProjectHeading}[2]{
    \vspace{-2pt}\item
    \begin{tabular*}{0.97\textwidth}{l@{\extracolsep{\fill}}r}
      \small#1 & #2 \\
    \end{tabular*}\vspace{-7pt}
}


\newcommand{\resumeOrganizationHeading}[4]{
  \vspace{-2pt}\item
    \begin{tabular*}{0.97\textwidth}[t]{l@{\extracolsep{\fill}}r}
      \textbf{#1} & \textit{\small #2} \\
      \textit{\small#3}
    \end{tabular*}\vspace{-7pt}
}

\newcommand{\resumeSubItem}[1]{\resumeItem{#1}\vspace{-4pt}}

\renewcommand\labelitemii{$\vcenter{\hbox{\tiny$\bullet$}}$}

\newcommand{\resumeSubHeadingListStart}{\begin{itemize}[leftmargin=0.15in, label={}]}
\newcommand{\resumeSubHeadingListEnd}{\end{itemize}}
\newcommand{\resumeItemListStart}{\begin{itemize}}
\newcommand{\resumeItemListEnd}{\end{itemize}\vspace{-5pt}}

%-------------------------------------------
%%%%%%  RESUME STARTS HERE  %%%%%%%%%%%%%%%%%%%%%%%%%%%%


\begin{document}

%---------- HEADING ----------

\begin{center}
    \textbf{\huge Thomas Gaviard} \\ \vspace{3pt}
    \small
    \faMobile \hspace{.5pt} +33 6 45 46 73 18
    $|$
    \faAt \hspace{.5pt} \href{mailto:thomas.gaviard@centrale.centralelille.com}{thomas.gaviard@centrale.centralelille.com}
    $|$
    \faGlobe \hspace{.5pt} \href{https://tms-gvd.github.io}{https://tms-gvd.github.io}
\end{center}



%----------- EDUCATION -----------

\section{Éducation}
  \vspace{3pt}
  \resumeSubHeadingListStart
    
    \resumeSubheading
      {Université de Lille
      % \normalfont{(Admission rate: 0.85\%)}
      }{Lille, France}
      {Master, Science des Données, cursus recherche}{Septembre 2022 \textbf{--} Avril 2024}
      \resumeItemListStart
        \resumeItem{Cours suivis : Math 1-2-3-4, Proba 1-2, Stats 1-2, ML 1-2-3-4, Deep Learning, AC 1-2}
        \resumeItem{Tous les cours sont dispensés en anglais ; Projet de recherche de 6 mois}
      \resumeItemListEnd
    
    \resumeSubheading
      {Centrale Lille
      % \normalfont{(Admission rate: 0.85\%)}
      }{Lille, France}
      {Ecole d'ingénieur généraliste}{Septembre 2019 \textbf{--} Août 2022}
      \resumeItemListStart
        \resumeItem{Cours suivis : Math, Physique, Programmation, Gestion de Projet}
        \resumeItem{Année de césure puis double-diplôme avec l'Université de Lille dans le Master Science des Données}
      \resumeItemListEnd
    
    \resumeSubheading
      {Lycée Louis-Le-Grand
      % \normalfont{(Admission rate: 0.85\%)}
      }{Paris, France}
      {CPGE PCSI-PC*}{Septembre 2016 \textbf{--} Juillet 2019}
  \resumeSubHeadingListEnd



%----------- RESEARCH EXPERIENCE -----------

\section{Expérience en recherche}
  \vspace{3pt}
  \resumeSubHeadingListStart
  
    \resumeSubheading
      {INRIA - équipe RAPSODI}{Lille, France}
      {Projet de Master, Optimisation et Calcul Numérique}{Octobre 2022 \textbf{--} Avril 2023}
      \resumeItemListStart
      \resumeItem{``Étude numérique d'un système dynamique de cellules de Voronoï et de leur limite continue", supervisé par Claire Chainais et Andrea Natale.}
      \resumeItem{Décomposition spatiale du domaine et résolution d'une EDO dont le potentiel est un problème d'optimisation convexe.}
    \resumeItemListEnd

    \resumeSubheading
      {INRIA - équipe MAGNET}{Lille, France}
      {Stage de césure, Machine Learning}{Mars 2022 \textbf{--} Août 2022}
      \resumeItemListStart
      \resumeItem{``Fairness in Federated Learning", supervisé par Michael Perrot.}
      \resumeItem{Elaboration d'une taxonomie détaillée des différents algorithmes d'apprentissage fédéré et \emph{fair}.}
      \resumeItem{Proposition et implémentation d'une nouvelle approche utilisant des techniques de pondération du gradient.}
    \resumeItemListEnd
      
  \resumeSubHeadingListEnd



%----------- WORK EXPERIENCE -----------

\section{Expérience professionnelle}
  \vspace{3pt}
  \resumeSubHeadingListStart
    
    \resumeSubheading
      {Euratechnologies}{Lille, France}
      {Stage, Data Scientist}{Septembre 2021 \textbf{--} Février 2022}
      \resumeItemListStart
          \resumeItem{3 projets en entreprise, encadrés par des professeurs de Centrale Lille, d'une durée de 2 mois chacun.}
          \resumeItem{\textbf{Projet 1 :}
          Détection de bots dans un jeu vidéo en ligne multijoueur par le biais de leur comportement.\\
          Traitement d'une grande quantité de données (50 Go) avec AWS Redshift.\\
          Implémentation d'un papier de recherche (\emph{Event2Vec} et \emph{Attention-based LSTM}).}
          \resumeItem{\textbf{Projet 2 :}
          Détection de défauts sur des rails de chemins de fer.\\
          Adaptation du modèle de détection d'objets Yolov5 et étude de systèmes experts.}
          \resumeItem{\textbf{Projet 3 :}
          Prévisions de ventes multivariées.\\
          Réalisation de statistiques exploratoires et implementation d'une méthode basée sur des réseaux récurrents.\\
          Présentation des résultats sur une interface web et déploiement d'une pipeline sur Google Cloud Platform.}
      \resumeItemListEnd
    
    \resumeSubheading
      {Helean}{Paris, France}
      {Stage, Data Scientist}{Juillet 2021 \textbf{--} Août 2021}
      \resumeItemListStart
          \resumeItem{Amélioration du modèle de prévision à l'aide de \emph{features engineering}.}
          \resumeItem{Enrichissement des données via \emph{webscraping}.}
      \resumeItemListEnd
    
  \resumeSubHeadingListEnd


%----------- SKILLS -----------

\section{Compétences}
  \vspace{2pt}
  \resumeSubHeadingListStart
    \small{\item{
        \textbf{Programmation :}{ Python, PostgreSQL, R, Matlab, \LaTeX} \\ \vspace{3pt}
        
        \textbf{Librairies :}{ numpy, pandas, matplotlib, scikit-learn, pytorch, tensorflow, cython, unit-testing} \\ \vspace{3pt}

        \textbf{Outils :}{ Git, GCP BigQuery, AWS Redshift} \\ \vspace{3pt}
        
        \textbf{Langues :}{ Français (langue maternelle), Anglais (C2), Espagnol (B2)}
        
        % \textbf{Frameworks}{: X, X, X} \\
        % \textbf{Developer Tools}{: X, X, X} \\
        % \textbf{Applications}{: X, X, X}
    }}
  \resumeSubHeadingListEnd



%----------- RELEVANT COURSEWORK -----------

% \section{Cours suivis}
%   \vspace{2pt}
%     \resumeSubHeadingListStart
%       \small{\item{Merci de consulter \href{https://tms-gvd.github.io}{ma page web} pour plus de détails.}}
%     \resumeSubHeadingListEnd


% ----------- HOBBIES -----------

\section{Loisirs}
  \resumeSubHeadingListStart
    \small{\item{Rugby, Musique, Poker}}
  \resumeSubHeadingListEnd



%-------------------------------------------
\end{document}
